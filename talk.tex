\documentclass{beamer}
\usepackage[utf8]{inputenc}

\usepackage{minted}
\usepackage{pgf}
\usepackage{tikz}
\usepackage{upquote}
\usepackage{hyperref}
\usetikzlibrary{arrows,automata}

\setbeamertemplate{footline}[frame number]
\setbeamertemplate{navigation symbols}{}

\title{A Dependently-Typed Zipper over GADT-Embedded ASTs}
\author{Donovan Crichton}
\date{November 2018}

\begin{document}
 
\frame{\titlepage}

\begin{frame}[fragile]
  \frametitle{Preliminaries}
  \begin{itemize}
    \item \textbf{Slides and examples available at:}
    \url{https://github.com/donovancrichton/talkdepzip.git}
  \item \textbf{About me:}
    \begin{itemize}
      \item Honours 'year' student at Griffith University.
      \item Working towards a type-correct genetic program through
              dependently-typed functional programming.
      \item About 18 months experience in FP, just under 12 with
              dependent types.
    \end{itemize}
  \end{itemize}
\end{frame}

\begin{frame}[fragile]
\frametitle{A refresher on dependent types 1.}
\begin{itemize}
  \item The most basic definition is a dependent data type (GADT in
          Haskell).
  \item Dependent data-types depend on being parameterised over a
        value for their construction.
  \item Distinguished from parameterised ADTs by the ability to
        specify the return type parameter of each data constructor.
\end{itemize}
\begin{minipage}{0.5\textwidth}
\begin{block}{A vector dependent on a length value.}
\begin{minted}[fontsize=\small,baselinestretch=1]{haskell}
  data Nat = Z | S Nat

  data Vec : (n : Nat) -> (e : Type) -> Type where
    Nil : Vec Z e
    (::) : (x : e) -> (xs : Vec n e) -> Vec (S Z) e
\end{minted}
\end{block}
\end{minipage}
\end{frame}

\begin{frame}[fragile]
  \frametitle{A refresher on dependent types 2.}

  \begin{block}{Why is this good?}
   If our length forms part of our type, we gain the ability
          to write correct functions with respect to vector length,
          without having to explicitly check. 
  \end{block}
  \begin{block}{Adding some vectors.}
  \begin{minted}[fontsize=\small]{haskell}
    -- total
    (+) : Num a => Vect n a -> Vect n a -> Vect n a
    (+) [] [] = []
    (+) (x :: xs) (y :: ys) = x + y :: xs + ys
  \end{minted}
  \end{block}
\end{frame}

\begin{frame}[fragile]
  \frametitle{A refresher on dependent types 3.}
  \begin{block}{$\Pi$ types.}
    \begin{itemize}
     \item The $\Pi$ type is a family of types that are indexed by a
           value (hence type families in Haskell).
     \item $\Pi$ types are used to calculate correct return types
             when given a specified value.
     \item In Idris $\Pi$ types only fully refine if the functions
             requiring them are marked as total.
     \item In Idris functions that return $\Pi$ types don't always refine in 
             function composition, recursive calls or let bindings.
     \end{itemize}
  \end{block}
\end{frame}

\begin{frame}[fragile]
  \frametitle{A refresher on dependent types 4.}
  \begin{block}{An example of using $\Pi$ types in Idris.}
  \begin{minted}[fontsize=\small]{haskell}
  Age : Type
  Age = Nat

  Name : Type
  Name = String

  data Material = Plastic | Wood | Metal | Cheese
  data Person = P Name Age
  data Object = O Material

  IsPerson : Bool -> Type
  IsPerson True = Person
  IsPerson False = Object

  isPerson : (x : Bool) -> IsPerson x
  isPerson True = P "Donovan Crichton" 33
  isPerson False = O Cheese
  \end{minted}
  \end{block}
\end{frame}

\begin{frame}[fragile]
  \frametitle{A refresher on dependent types 5.}
  \begin{block}{$\Sigma$ types.}
  \begin{itemize}
    \item $\Sigma$ types are a pairing of a value, and a type that
            depends on that value (They are also called dependent
                  pairs).
    \item $\Sigma$ types are useful when you want some basic type
            inference around dependent types!
    \item In Idris it is difficult to extract either size of a $\Sigma$
          type pair. Particularly when that pair is under a
          constructor.
    \item Due to the dependent nature of the
            \mintinline{haskell}{fst} and \mintinline{haskell}{snd}
            functions on dependent pairs, they cannot be used with
            ordinary maps, folds, binds etc.
  \end{itemize}
  \end{block}
\end{frame}

\begin{frame}[fragile]
  \frametitle{A refresher on dependent types 6.}
  \begin{block}{An example using $\Sigma$ types in Idris.}
  \begin{minted}[fontsize=\small]{haskell}
  data DPair : (a : Type) -> (P : a -> Type) -> Type where
    MkDPair : (x : a) -> (pf : P x) -> DPair a P

  -- also has some syntactic sugar in Idris.
  f : (x : Bool ** IsPerson x)
  f = (_ ** isPerson True)

  g : Num a => (n : Nat ** Vec n a)
  g = (_ ** [1, 2, 3])

  -- this is particularly useful if we are passing a vector
  -- of unknown length in as an argument.
  len : Num a => Vec n a -> (x : Nat ** Vec x a)
  len x = (_ ** x)

  -- len [1, 2, 3] returns
  -- (3 ** [1, 2, 3]) : (x : Nat ** Vec x Integer)
  \end{minted}
  \end{block}
\end{frame}

\begin{frame}[fragile]
  \frametitle{A refresher on dependent types 7.}
  \begin{block}{Why are $\Sigma$ and $\Pi$ good?}
  \begin{itemize}
    \item It turns out that the curry-howard isomorphsim
      still holds with the introduction of dependent types.
    \item Lets say we have some proof $P$ $x$, using a
      dependent-data-type or a $\Pi$ type is saying $\forall x, P$ $x$.
      Using a $\Sigma$ type is saying $\exists x, P$ $x$.
    \item This brings the isomorphic logic from propositional logic
      to first-order or predicate logic, allowing us to write proofs
      in our code!
  \end{itemize}
  \end{block}
\end{frame}

\begin{frame}[fragile]
  \frametitle{Dependent data-type embedded DSLs}
  \begin{block}{test}
  \end{block}
\end{frame}
\end{document}
